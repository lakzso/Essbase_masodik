\chapter{Implementáció}\label{sect:Ellaboration}
 %Xtextet csináltam ; osztálydiagram a lényeges cuccokkal; hogyan teszteltem,
%mérési eredmények
\section{Xtext metamodell}
Az új lekérdező nyelv fejlesztése során először is létrehoztam egy metamodellt Xtext-ben, ahol megadtam hogy hogyan épül föl Essbase lekérdezés, milyen parancsokból áll, közöttük milyen logika van. Mindezt azért hogy az Xtext a már kész lekérdezést majd tudja validálni.

A metamodell alapján egy Essbase lekérdező nyelv alapvetően Query-kből, valamint a riport kimenetének megadásából (Reports-ból áll):

\begin{lstlisting}[language=java,morekeywords={generate,dim,group,row,link,reportParameter,report,query},alsoletter={-},breaklines=true]
Model:
	(Queries+=Query)*
	(Reports+=Report)*
	;
\end{lstlisting}

Riport kimenet megadásakor a 'report' kulcsszó után megadunk egy olyan Query referenciát, amit már előzőleg definiáltunk.

\begin{lstlisting}[language=java,morekeywords={generate,dim,group,row,link,reportParameter,report,query},alsoletter={-},breaklines=true]
Report:
	'report' {Report} repout=[Query]
;
\end{lstlisting}

Definiálom hogy hogyan épül fel egy Query, adok neki egy azonosítót (ID) és előírom hozzá a paramétereket, amiket egy Queryba meg lehet adni:

\begin{lstlisting}[language=java,morekeywords={generate,dim,group,row,link,reportParameter,report,query},alsoletter={-},breaklines=true]
Query:
	name=ID '=' 'query' '{'
	InitialStatement
	(Query+=ReportQueryParameters)* '}'
	;

ReportQueryParameters:
	Column | Row | Descendants | Declaration | Reference | Child | Link | ReportParameter;
	
\end{lstlisting}

Létrehozok egy öröklési hierarchiát, hogy a dimenzió, csoport és tag deklarációkat hasonlóan kezelje a program.
\begin{lstlisting}[language=java,morekeywords={generate,dim,group,row,link,reportParameter,report,query},alsoletter={-},breaklines=true]

Descendants:
	'descendants' group=ID ',' dimension=Reference;

Declaration:
	DimensionDeclaration | GroupDeclaration | MemberDeclaration;
\end{lstlisting}


A következőkben definiálok objektum referenciákat, amik az Essbase kockában is vannak, ezen itt megadott objektmokra később lehet hivatkozni, elkerülve az esetleges elírásokat. Ilyen objektumok a

\begin{itemize}
  \item dimenziók
\begin{lstlisting}[language=java,morekeywords={generate,dim,group,row,link,reportParameter,report,query},alsoletter={-},breaklines=true]

DimensionDeclaration:
	'dim' name=ID value=STRING;

\end{lstlisting}
  \item csoportok
\begin{lstlisting}[language=java,morekeywords={generate,dim,group,row,link,reportParameter,report,query},alsoletter={-},breaklines=true]
  
GroupDeclaration:
	'group' name=ID value=STRING;
\end{lstlisting}
  \item tagok
\begin{lstlisting}[language=java,morekeywords={generate,dim,group,row,link,reportParameter,report,query},alsoletter={-},breaklines=true]

MemberDeclaration:
	'member' name=ID value=STRING;

\end{lstlisting}

\end{itemize}
Végül definiálok olyan parancsokat, amibe meg lehet adni, hogy a visszakapott mátrixba milyen sorok(row) és oszlopok(column) szerepeljenek, valamint a hierarchia milyen szintjéig jelenjenek meg a tagok(link):
\begin{lstlisting}[language=java,morekeywords={generate,dim,group,row,link,reportParameter,report,query},alsoletter={-},breaklines=true]

Row:
	'row' {Row} '{' dimensions+=Reference (',' dimensions+=Reference)* '}';

Column:
	'column' {Column} '{' dimensions+=Reference (',' dimensions+=Reference)* '}'
; 	
Link:
	'link' {Link} '{' desc=Reference ',' lev=Reference '}'
; 	

\end{lstlisting}
% \begin{lstlisting}[language=java,morekeywords={generate,dim,group,row,link,reportParameter,report,query},alsoletter={-},breaklines=true]
% 
% Child:
% 	'child' {Child} '{' dimensions+=Reference (',' dimensions+=Reference)* '}'
% ; 
% \end{lstlisting}
% 
% \begin{lstlisting}[language=java,morekeywords={generate,dim,group,row,link,reportParameter,report,query},alsoletter={-},breaklines=true]
% 	
% ReportParameter:
% 	'reportParameter' {ReportParameter} '{' reparam=StringReference '}'
% ;
% \end{lstlisting}
% 
% 
% \begin{lstlisting}[language=java,morekeywords={generate,dim,group,row,link,reportParameter,report,query},alsoletter={-},breaklines=true]
% 
% Reference:
% 	referred=[Declaration] | StringReference;
% 
% \end{lstlisting}
% 
% \begin{lstlisting}[language=java,morekeywords={generate,dim,group,row,link,reportParameter,report,query},alsoletter={-},breaklines=true]
% 
% StringReference:
% 	value=STRING;
% \end{lstlisting}


\section{Xtext példa lekérdezés}

Az Xtext metamodell megírás után készítettem saját példa lekérdezést is, amit az
Xtext már tud validálni. A példámban egy általános lekérdezést fogalmazok meg a
kocka 2 dimenziójáról (Költségnem és Objektum), valamint 2 csoportjáról
(Objektumok, és Költségnemek), ezeket az adatokat lekérdezzük a kockából, majd
latex pdf-be aggregálva egy diagramon megjelenítjük őket.

Példa lekérdezés:
\begin{lstlisting}[language=java,morekeywords={dim,group,row,link,reportParameter,report,query},alsoletter={-},breaklines=true]
SajatQuery1=query {
	{SUPEMPTYROWS}
{DECIMAL 10}
{TABDELIMIT}
{ROWREPEAT}
{SUPBRACKETS}
{SUPCOMMAS}
{NOINDENTGEN}

dim ktg "Koltsegnem"
dim obj "Objektum"
group objects "Objektumok"
group ktgs "Koltsegnemek"

row {obj,ktg}
link {objects,obj}
link {ktgs,ktg}
reportParameter {"UGYF"}   }
report SajatQuery1
\end{lstlisting}