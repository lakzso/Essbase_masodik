\chapter{Implementáció}\label{sect:Ellaboration}
 %Xtextet csináltam ; osztálydiagram a lényeges cuccokkal; hogyan teszteltem,
%mérési eredmények
\section{Xtext metamodell}
Az új lekérdező nyelv fejlesztáse során először is létrehoztam egy metamodellt
Xtext-ben, ahol megadtam hogy hogyan épül föl Essbase lekérdezés, milyen
parancsokból áll, közöttük milyen logika van, erre azért van szükség hogy az
Xtext a már kész lekérdezést majd tudja validálni.

A metamodell alapján egy Essbase lekérdező nyelv alapvetően Query-ből,
valamint a riport kimenetének megadásából (Reports-ból áll). A Query-n belűl
kell megadni a lekérdezés részleteit, mennyire szeretnénk aggregálva látni az
adatokat, valamint melyik diemnzót szerernénk látni.

Xtext metamodell:
\begin{lstlisting}[language=java,morekeywords={generate,dim,group,row,link,reportParameter,report,query},alsoletter={-},breaklines=true]

Model:
	(Queries+=Query)*
	(Reports+=Report)*
	;
Query:
	name=ID '=' 'query' '{'
	InitialStatement
	(Query+=ReportQueryParameters)* '}'
	;

Report:
	'report' {Report} repout=[Query]
;

ReportQueryParameters:
	Column | Row | Descendants | Declaration | Reference | Child | Link | ReportParameter;

InitialStatement:
"{SUPEMPTYROWS}
{DECIMAL 10}
{TABDELIMIT}
{ROWREPEAT}
{SUPBRACKETS}
{SUPCOMMAS}
{NOINDENTGEN}";

Descendants:
	'descendants' group=ID ',' dimension=Reference;

Declaration:
	DimensionDeclaration | GroupDeclaration | MemberDeclaration;

DimensionDeclaration:
	'dim' name=ID value=STRING;

GroupDeclaration:
	'group' name=ID value=STRING;

MemberDeclaration:
	'member' name=ID value=STRING;

Row:
	'row' {Row} '{' dimensions+=Reference (',' dimensions+=Reference)* '}';

Column:
	'column' {Column} '{' dimensions+=Reference (',' dimensions+=Reference)* '}'
; 	
Link:
	'link' {Link} '{' desc=Reference ',' lev=Reference '}'
; 	
Child:
	'child' {Child} '{' dimensions+=Reference (',' dimensions+=Reference)* '}'
; 	
ReportParameter:
	'reportParameter' {ReportParameter} '{' reparam=StringReference '}'
;

Reference:
	referred=[Declaration] | StringReference;

StringReference:
	value=STRING;
\end{lstlisting}

\section{Xtext példa lekérdezés}

Az Xtext metamodell megírás után készítettem saját példa lekérdezést is, amit az
Xtext már tud validálni. A példámban egy általán lekérdezést fogalmazok meg a
kocka 2 dimenziójáról (Költségnem és Objektum), valamint 2 csoportjáról
(Objektumok, és Költségnemek), ezeket az adatokat lekérdezzük a kockából, majd
latex pdf-be aggregálva egy diagramon megjelenítjük őket.

Példa lekérdezés:
\begin{lstlisting}[language=java,morekeywords={dim,group,row,link,reportParameter,report,query},alsoletter={-},breaklines=true]
SajatQuery1=query {
	{SUPEMPTYROWS}
{DECIMAL 10}
{TABDELIMIT}
{ROWREPEAT}
{SUPBRACKETS}
{SUPCOMMAS}
{NOINDENTGEN}

dim ktg "Koltsegnem"
dim obj "Objektum"
group objects "Objektumok"
group ktgs "Koltsegnemek"

row {obj,ktg}
link {objects,obj}
link {ktgs,ktg}
reportParameter {"UGYF"}   }
report SajatQuery1
\end{lstlisting}