%----------------------------------------------------------------------------
\chapter*{\bevezeto}\addcontentsline{toc}{chapter}{\bevezeto}
%----------------------------------------------------------------------------
Napjainkban a vállalatoknál egyre több adattal dolgoznak. Ezeket
tipikusan adattárházba mentik, majd erre OLAP(TODO) adatbázist építenek annak
érdekében az adatok összefüggéseit lehessen vizsgálni. Az adatok összefüggéseit
ti olyan lekérdezéseket megfogalmazni, amivel az adatokat
összevontan lehet megjeleníteni (például összegezni vagy átlagolni).
Ilyen OLAP adatbázis például, a gyakorlatban igen sokszor használt és elterjedt
adatbázis, az Essbase.

Az előzőekben említett lekérdezések megfogalmazásánál 
nehézséget okoz fejlesztéskor, hogy (i) a fejlesztő környezet nem validálja a megírt kódot, (ii) mivel a nyelv deklaratív, nem lehet benne hibát keresni, (iii) hibáknál gyakran üres megoldást kapunk. Ezen kívűl a fejlesztő környezet is kevés segítséget nyújt, mert a lekérdezésekkor egy egyszerű szöveg fájlt kell megfogalmazni.
Továbbá az eredmény halmaz egy n-dimenziós mátrix, ami átláthatatlan és nehéz feldolgozni. Ezeken
kívűl, mivel kétféle különböző funkcionalitással bíró lekérdező nyelv van az Essbase-hez, sokszor előforduló jelenség 
,hogy ezeket kombinálva kell használni, ami nehézkes ami további nehézségeket okoz. Ilyen probléma
például a költségfelosztás, amikor egyik memberben vagy csoportban lévő
költséget (az eltárolt számot), két másik memberre vagy egy egész csoportra
szeretnénk valahogy felosztani valamilyen, arány szerint.

Ezeket a problémákat szeretném kezelni dolgozatomban, mégpedig úgy hogy saját lekérdező nyelvet és hozzá szerkesztőt készítek, ami egyrészt megoldást kínál a felvetett problémákra, valamint kombinálja a kétféle lekérdező
nyelv (Riport, MDX) előnyeit egybe. Célkitűzésem tehát egy magas szintű jelentés
és lekérdező nyelv készítése, ami ezen problémákra választ ad.

Dolgozatomban bemutatok egy olyan Xtext alapú nyelvet, amely segítségéveé le lehet kérdezni az
Essbase adatbázisból, valamint hogy ezt a gyakorlatba is lehessen használni,
készítek hozzá egy szerkesztőt Eclipse alapú szöveges szerkesztőt, ahol a megírt lekérdezést
lehet futtatni. Végül a lekérdezések eredményét Latex segítségével egy PDF dokumentumba ábrázolom, amely testzőlegesen szempléltetheti a megoldást.

Mindezek alapján a fejlesztés sokkal gyorsabb és hatékonyabb lesz, mert nem kell
kétféle programozási nyelvet használni, hatékonyabban, kevesebb munkával meg
lehet ugyanazokat, vagy még bonyolultabb Essbase lekérdezéseket vagy speciális
műveleteket írni.



 
