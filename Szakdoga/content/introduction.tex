%----------------------------------------------------------------------------
\chapter*{\bevezeto}\addcontentsline{toc}{chapter}{\bevezeto}
%----------------------------------------------------------------------------
Napjainkban, a vállalatoknál egyre több adattal dolgoznak, ezeket adattárházba mentik, 
majd erre OLAP adatbázist építenek, hogy az adatok összefüggéseit lehessen vizsgálni, 
azáltal hogy olyan lekérdezéseket lehet benne megfogalmazni, amivel az adatokat összevontan lehet nézni. 
Ilyen például, a gyakorlatban igen sokszor használt és elterjedt adatbázis, az Essbase.

Az előzőekben említett lekérdezések megfogalmazásánál, ha például egy nagyobb OLAP kockáról van szó, 
nehézséget okozhat fejlesztéskor, hogy egy sima szöveges fájlt szerkesztünk, 
ebből következően nem lehet benne hibát keresni, objektumok beírásánál a
fejlesztő környezet nem tudja azt validálni, ebből következően gyakran a
lekérdezéseknél üres megoldást kapunk, valamint az eredmény halmazt egy
n-dimenziós mátrixba kapjuk, ami átláthatatlan, és nehéz feldolgozni. Valamint
ezeken kívűl, sokszor előforduló jelenség hogy mivel kétféle lekérdező nyelv van
az Essbase-hez, különböző funkcionalitással, ezeket bizonyos problémáknál
kombinálva kell használni, ami nehézkes, és átláthatatlan. Ilyen probléma
például a költségfelosztás, amikor egyik memberben vagy csoportban lévő
költséget (az eltárolt számot), két másik memberre vagy egy egész csoportra
szeretnénk valahogy felosztani valamilyen, arány szerint.

Ezeket a problémákat szeretném kezelni dolgozatomban, mégpedig úgy hogy Xtext
metamodellező nyelvben implementálok saját lekérdező nyelvet, ami egyrészt
megoldást kínál a felvetett problémákra, valamint kombinálja a kétféle lekérdező
nyelv (Riport, MDX) előnyeit egybe. Célkitűzésem tehát egy magas szintű jelentés
és lekérdező nyelv készítése, ami ezeket tudja.

Dolgozatomban bemutatok egy olyan konkrét Xtext nyelvet ami le tud kérdezni az
Essbase adatbázisból, valamint hogy ezt a gyakorlatba is lehessen használni,
készítek hozzá egy szerkesztőt Eclipse környezetben, ahol a megírt lekérdezést
lehet futtatni és egy latex Pdf ennek az eredményét ki lehet riportálni további
felhasználás céljából.

Mindezek alapján a fejlesztés sokkal gyorsabb és hatékonyabb lesz, mert nem kell
kétféle programozási nyelvet használni, hatékonyabban, kevesebb munkával meg
lehet ugyanazokat, vagy még bonyolultabb Essbase lekérdezéseket írni.


 
