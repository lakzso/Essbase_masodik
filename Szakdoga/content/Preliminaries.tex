%----------------------------------------------------------------------------
\chapter{Alapfogalmak}\label{sect:Preliminaries}
%----------------------------------------------------------------------------
\section{Essbase adatbázis}
Alapvetően kétféle adatbázis létezik lekérdezés szempontjából az OLTP (On Line
Transaction Processing, online tranzakciófeldolgozás), ahol sok tranzakció van és ezek kis
módosításokat (UPDATE,INSERT) vagy lekérdezéseket csinálnak az
adatbázisban és az OLAP (On Line Analitical Processing vagy online
analitikai feldolgozás), ami kimondottan a lekérdezésekre lett optimalizálva,
jelentések készítésére. Például hogy egy autó hova ment, közbe mennyi volt a
fogyasztása ezt tipikusan OLTP adatbázisba szeretnénk tárolni, mert sok
tranzakció keletkezik sok INSERT-el. Viszont ha van pár céges autónk egy vállalatnál, ezeknek az autóknak vannak kölségeik, 
benzin, amortizáció, de ezek a költségek az OLTP adatbázisunkba vannak soronként
egy táblába, egy adott szervezeti egységhez rendelve és szeretnénk tudni azt
hogy aggregálva az autóknak mennyi volt a költségeik anélkül hogy egy ilyen
lekérdezés órákig tartana (az egész táblát össze kéne adni majd egyenként
leosztani valamilyen szempont alapján), ehhez már árdemes felépíteni egy OLAP
kockát, ami automatikusan képes az ilyen jellegű aggregációra és arra is hogy az
szervezeti egység költségeit el tudja osztani minden egyes autóra, vagy akár más
objektumra, például szolgáltatásokra, költséghelyekre.

\section{Xtext keretrendszer}
Az Xtext gyakorlatilag egy olyan keretrendszer, amibe lehet definiálni egy
saját, domén specifikus programozási nyelvet, metamodellt. Ha ez kész, akkor
ebben a programozás nyelvben lehet utána fejleszteni, majd a kódot ezt a kódot az Xtext tudja validálni, 
és feldolgozni, ezeket az adatokat objekumokba megkapjuk és java-ban ezt utána
fel lehet dolgozni.

\section{Eclipse plugin}
Eclipse környezetben lehet készíteni saját beépülő modulokat, amik miután
föltelepültek gyakorlatilag teljes testre tudják szabni az Eclipse-t,
dolgozatomban én is készítek egy saját plugint, szerkesztőt, amibe lehet
szerkeszteni a létrehozott programozási nyelv kódját.

\section{Latex és riportolási technikák}
Megjelenítésre, riport kimenetnek a dolgozatomban én latexet választottam, mert
a forráskódja java kódból generálható és vannak olyan grafikus megjelenítő
csomagjai, amivel az Essbase-ből kijövő nyers adatokat meg lehet jeleníteni.

\section{Xtend}
Sok új nyelvi elemmel rendelkező java továbbfejlesztése. Hasznos funkciója hogy
lehet olyan sablont készíteni, amiből az Xtend java osztályokat generál
StringBuilder-ek segítségével és egyszerű paraméterezéssel meg lehet adni a
sablonnak a változó értékeket.

