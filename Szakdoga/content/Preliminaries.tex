%----------------------------------------------------------------------------
\chapter{Esettanulmány}\label{sect:Preliminaries}
%----------------------------------------------------------------------------
\section{Essbase adatbázis}
Alapvetően kétféle adatbázis létezik lekérdezés szempontjából az OLTP (On Line
Transaction Processing, online tranzakciófeldolgozás), ahol sok tranzakció van és ezek kis
módosításokat (UPDATE,INSERT) vagy lekérdezéseket csinálnak az
adatbázisban és az OLAP (On Line Analitical Processing vagy online
analitikai feldolgozás), ami kimondottan a lekérdezésekre lett optimalizálva,
jelentések készítésére. Például egy számla vezető rendszerbe a napi
tranzakciókat és ezek könyvelését OLTP-be célszerű tárolni, ezzel szemben azt
hogy a számla havi forgalma milyen volt aggregálva szeretnénk látni, ehhez már
érdemes OLAP adatbázist használni, mert az ilyen jellegű lekérdezések OLTP-be
nagyon lassúak. 
