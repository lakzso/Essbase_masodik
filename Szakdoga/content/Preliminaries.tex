%----------------------------------------------------------------------------
\chapter{Alapfogalmak}\label{sect:Preliminaries}
%----------------------------------------------------------------------------
\section{Essbase adatbázis}
Alapvetően kétféle adatbázis létezik lekérdezés szempontjából az OLTP (On Line
Transaction Processing, online tranzakciófeldolgozás), ahol sok tranzakció van és ezek kis
módosításokat (UPDATE,INSERT) vagy lekérdezéseket csinálnak az
adatbázisban és az OLAP (On Line Analitical Processing vagy online
analitikai feldolgozás), ami kimondottan a lekérdezésekre lett optimalizálva,
jelentések készítésére. Például hogy egy autó hova ment, közbe mennyi volt a
fogyasztása ezt tipikusan OLTP adatbázisba szeretnénk tárolni, mert sok
tranzakció keletkezik sok INSERT-el. Viszont ha van pár céges autónk egy vállalatnál, ezeknek az autóknak vannak kölségeik, 
benzin, amortizáció, de ezek a költségek az OLTP adatbázisunkba vannak soronként
egy táblába, egy adott szervezeti egységhez rendelve és szeretnénk tudni azt
hogy aggregálva az autóknak mennyi volt a költségeik anélkül hogy egy ilyen
lekérdezés órákig tartana (az egész táblát össze kéne adni majd egyenként
leosztani valamilyen szempont alapján), ehhez már árdemes felépíteni egy OLAP
kockát, ami automatikusan képes az ilyen jellegű aggregációra és arra is hogy az
szervezeti egység költségeit el tudja osztani minden egyes autóra, vagy akár más
objektumra, például szolgáltatásokra, költséghelyekre.

\section{Xtext keretrendszer}

\section{Eclipse plugin}

\section{Latex és riportolási technikák}