%----------------------------------------------------------------------------
\chapter{Előismeretek}\label{sect:Preliminaries}
%----------------------------------------------------------------------------
\section{Essbase adatbázis}
Alapvetően kétféle adatbázis típus létezik lekérdezés szempontjából. Az egyik az OLTP (On Line Transaction Processing, online tranzakciófeldolgozás), ahol sok tranzakció van és ezek kis módosításokat (UPDATE,INSERT), vagy kisebb, rövidebb ideig tartó lekérdezéseket végeznek el az adatbázisban. A másik az OLAP (On Line Analitical Processing vagy online analitikai feldolgozás), ami kimondottan a lekérdezésekre lett optimalizálva, jelentések készítésére. Ilyen adatbázis például az Essbase.


\subsection{OLAP kocka felépítése}

\subsection{Essbase lekérdezések}

\subsection{Essbase esettanulmány}
Tegyük fel, hogy van egy céges autónk és szeretnénk eltárolni az autó forgalmi adatait, hová ment, közbe mennyi volt a fogyasztása. Ezt tipikusan OLTP adatbázisba érdemes tárolni, mert sok tranzakció keletkezik, sok beszúrás művelettel. Ellenben tegyük fel, hogy több céges autónk van és az autókra keletkeznek kölségek, például benzin, amortizáció. Valamint tételezzük föl hogy ezekből a költségekből van olyan, ami az egész szervezeti egységhez kötődik, például a benzint egyszerre vesszük meg és mindenki tankol ebből annyit amennyire szüksége van. Ezek a költségek az OLTP adatbázisunkba vannak eltárolva egy táblába autókra és szervezeti egységekre külön. Ha megvannak ezek az adataink, akkor gyakran felmerülő igény, hogy le szeretnénk kérdezni valamilyen szempont szerint aggregálva ezeket a költségeket. Például összeadni vagy átlagolni hogy egy autóra és szervezeti egységre mennyi költség keletkezett. Egy ilyen lekérdezés, nagy tábla esetén sokáig tart, mert a tábla összes során végig kell menni.  Ebben az esetben már érdemes felépíteni egy OLAP kockát, erre az OLTP adatbázisra, ami automatikusan képes az ilyen jellegű aggregációra, összeadásra, átlagolásra stb. Ezen kívűl képes arra is, hogyha keletkezik olyan költség ami egy szervezeti egységhez köthető (az előzőekben említett benzin amit egyszerre veszünk), azt le tudja osztani minden egyes autóra lebontva, például megtett út alapján.

\section{Xtext keretrendszer}
Az Xtext egy olyan keretrendszer, amiben saját szakterület specifikus programozási nyelvet lehet definiálni. Ha ez kész, akkor ebben a programozás nyelvben lehet utána fejleszteni, majd ezt a kódot az Xtext tudja validálni és feldolgozni, ezeket az adatokat objekumokba megkapjuk és java-ban ezt utána fel lehet dolgozni.

\section{Eclipse plugin}
Az Eclipse fejlesztő környezet egy modulokkal (úgynevezett pluginekkel) bővíthető keretrendszer, melyek segítségével saját funkciókkal szabhatjuk testre a környezetet. Önálló laboratórium során, én is készítek egy saját plugint, szerkesztőt, amibe a létrehozott programozási nyelv kódját lehet szerkeszteni.

\section{Latex és riportolási technikák}
Megjelenítésre, riport kimenetnek a dolgozatomban én latexet választottam, mert a forráskódja java kódból generálható és vannak olyan grafikus megjelenítő csomagjai, amivel az Essbase-ből kijövő nyers adatokat meg lehet jeleníteni.

\section{Xtend}
Sok új nyelvi elemmel rendelkező java nyelv továbbfejlesztése. Hasznos funkciója, hogy lehet olyan szöveg sablont készíteni, amiből az Xtend java osztályokat generál StringBuilder-ek segítségével és egyszerű paraméterezéssel meg lehet adni a sablonnak a változó értékeket.

