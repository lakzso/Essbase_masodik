%----------------------------------------------------------------------------
\chapter{Összefoglalás és jövőbeli tervek}\label{sect:Summary}
%----------------------------------------------------------------------------
%Eredmények -elmélet, gyakorlati pl plusz validációs szabályok
  %         -gyakorlati eredmény 
   %        -milyen felhasználási eredmény
Összefoglalva jelenleg a fejlesztés a következő állapotban tart:
\begin{itemize}
  \item elkészült a lekérdező nyelv metamodellje
  \item elkészült a lekérdező nyelv szerkesztője Eclipse környezetben
  \item validálom a lekérdezést már íráskor, az Xtext metamodell segítségével
  \item az Essbase-től kapott adatokból, Xtend sablonokkal Latex dokumentumokat és diagramokat generálok
\end{itemize}

A jövőben ezeket szeretném bővíteni extra validációs szabályokkal, valamint szeretném az MDX és a Riport előnyeit összetenni az új lekérdező nyelbve, ebből kifolyólag olyan domén specifikus problémákat is meg lehet majd oldani, amelyeket idáig nem lehetett csak mindkét nyelv együttes felhasználásával. Ilyen probléma például a költségfelosztás, ami azt jelenti hogy egy dimenzióról vagy egy csoportról szeretnénk átteni, a költségeket egy másik dimenzióra, vagy csoportra valamilyen arány szerint, ezt nyelvi elemként szeretném majd a jövőbe támogatni.
